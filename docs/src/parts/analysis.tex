\section{Analyse}
\label{sec:Analysis}


\subsection{Netzwerkeigenschaften}
\label{sec:overview-numbers}
Als erstes soll die Art des Netzwerkes analysiert werden. Der Begriff \guillemotleft Hub\guillemotright wird sehr
häufig im Zusammenhang mit Luftfahrt und Flughäfen verwendet.
Deshalb wird auch erwartet, dass das Netzwerk ein natürliches bzw. skalenfreies Netzwerk bildet.
Zufällige Netzwerke haben eine gleichmässige Gradverteilung während die Verteilung bei natürlichen Netzwerken einem
Potenzgesetz folgt.
Das Histogramm der Knotengrade deutet sehr stark auf eine exponentielle Verteilung hin (siehe Abb. \ref{fig:degreeHistogram}).

\begin{figure}
    \centering
    \includegraphics[width=0.75\linewidth]{images/degree-histogram.pdf}
    \caption{Histogramm der Grade.}
    \label{fig:degreeHistogram}
\end{figure}

\subsection{Beschaffenheit}

Die Daten basieren auf dem Datensatz für die Flughäfen vom 8. Oktober 2018 sowie Flugverbindungen vom 9. November 2018.
\vspace{2em}

\begin{itemize}
    \item Anzahl Flughäfen: 9448
    \item Anzahl Routen: 180529
    \item Netzwerkdurchmesser: 10
    \item Durchschnittliche Pfadlänge: 3.82
    \item Durchschnittler Grad $\langle k\rangle$: 38.215
    \item Zweites Moment $\langle k^{2}\rangle$: 35676.88
    \item Exponent $\gamma$ 2.24
\end{itemize}

\subsubsection{Annäherung \gamma-Exponent}
Der grosse Unterschied zwischen $\langle k \rangle \text{sowie} \langle k^{2} \rangle$ deutet auf eine starke Streuung der
Grade und somit auf ein skalenfreies Netzwerk hin.
Gemäss des Algorithmus in Network Science\cite{barabasi-network-science} soll versucht werden, den Gamma-Exponent für das Flugroutennetzwerk in Annahme einer Potenzverteilung anzunähern.

Zur Schätzung des $ \gamma$- Exponenten wird die Maximum Likelihood Methode wie in \cite{barabasi-network-science} und näher
in \cite{clauset-power-law-distribution} beschrieben.

$$ \gamma = 1 + N\Bigg[\sum_{i=1}^{N}ln \frac{k_{i}}{K_{min} - \frac{1}{2}}\Bigg]^{-1} $$

Das anschliessende Fitting mittels kumulierter Verteilungsfunktion siedelt den $\gamma$-Exponenten bei 2.24 an.
In Abbildung \ref{fig:degreeDist} wird die empirische Verteilung sowie die Verteilung gemäss Exponenten gezeigt.
In der Log-Skala ist gut sichtbar, dass es sich bei der Verteilung nicht um eine reine Potenz-Verteilung handeln kann,
da sie eine höhere \guillemotleft Sättigung\guillemotright im unteren Gradbereich aufweist (\guillemotleft Low Degree Saturation\guillemotright).
Zudem ist die starke Streuung der Verteilung bei höheren Graden deutlich sichtbar.
Weshalb die Verteilung nach Gamma-Exponent ausserhalb der empirischen Verteilung liegt ist unklar, womöglich ist dies aber auf einen
Berechnungsfehler zurückzuführen.

\begin{figure}
    \centering
    \includegraphics[width=0.75\linewidth]{images/degree-distribution.pdf}
    \caption{Gradverteilung auf der Log-Skala}
    \label{fig:degreeDist}
\end{figure}


\subsubsection{Welches sind die wichtigsten Flughäfen?}
Wie wird ein Flughafen als wichtig klassifiziert?
Wichtig ist ein Flughafen sicher dann, wenn er einerseits ein Hub ist, also viele Verbindungen zusammenführt und wieder verteilt.
Ausserdem ist ein Flughafen wichtig, wenn er als Brücke zwischen mehreren Teilnetzen agiert.
Diese Eigenschaft ist definiert in der sog. \guillemotleft Betweenness Centrality\guillemotright.
Je mehr kürzeste Pfade über einen bestimmten Flughafen führen, desto höher ist dieser Zentralitätswert.

\begin{table}
    \centering

\begin{tabular}{l l c r}
ICAO & Stadt & Betweenness & k \\ \hline
EDDF & Frankfurt & 0.007061 & 2359 \\
OMDB & Dubai & 0.007035 & 2092 \\
LFPG & Paris & 0.007003 & 2658 \\
LTBA & Istanbul & 0.006689 & 2044 \\
ZBAA & Peking & 0.006433 & 2156 \\
KLAX & Los Angeles & 0.005778 & 2764 \\
PANC & Anchorage & 0.005543 & 389 \\
KORD & Chicago & 0.004787 & 3438 \\
CYYZ & Toronto & 0.004419 & 2299 \\
YSSY & Sydney & 0.004110 & 1445 \\

\end{tabular}

    \caption{Flughäfen mit den höchsten Betweenness Zentralitätsmassen}
    \label{tbl:between}
\end{table}

\subsubsection{Wie ist die Beschaffenheit des globalen Flughafennetzwerkes im Bezug auf die Robustheit?}
Begründung: Im Flugnetz finden täglich bis zu 200000 Flugbewegungen statt\footnote{http://www.reisereporter.de/artikel/4533-rekord-tag-luftfahrtgeschichte-flugzeuge-und-fluege-pro-tag-weltweit-flug-tracker-flightradar-misst-verkehrsreichsten-tag}.
Für diese enorme Zahl finden erstaunlich wenige Störungen und Ausfälle statt.
Aber wie gut ist das Netzwerk aufgebaut?
Wieviele Flughäfen müssten ausfallen, damit das Gesamtnetz in einzelne isolierte Unternetze zerfällt und damit der weltweite Luftverkehr zum erliegen kommt?

Der kritische Anteil an geschlossenen Flughäfen, der nötig ist, damit das Gesamtnetzwerk in mehrere kleinere, nicht zusammenhängende Netzwerke zerfällt lässt sich wie folgt berechnen:

$$ f_c = 1 - \frac{1}{\frac{\langle k^{2}\rangle}{\langle k\rangle} - 1} = 0.9989 $$

Vom Flugnetzwerk müssten also über 99\% der Flughäfen zufällig geschlossen werden, bevor das Netzwerk in mehrere kleinere Netzwerke zerfällt.


\subsection{Verspätungen}
Mobilität per Flugzeug wächst seit Beginn der Luftfahrt.
In einem kürzlich erschienenen Artikel spricht die NZZ gar von einem jährlichen Wachstum der Anzahl Flüge von 4\%\footnote{https://www.nzz.ch/wirtschaft/wieso-in-europa-viel-zu-viele-fluege-verspaetet-sind-ld.1410842}.
Im selben Artikel werden die Auswirkungen dieses immer dichter werdenden Verbindungsnetzwerks dargelegt: Häufigere und längere Verspätungen.
Ursachen für Verspätungen sind z.B. Streiks, Wetterbedingungen, Verkehrsüberlastungen, aber auch ineffiziente Arbeitsabläufe während des sog. \guillemotleft Turnarounds\guillemotright,
die Zeit, während der das Flugzeug geparkt ist und entladen, betankt und wieder beladen bzw. für den nächsten Flug vorbereitet wird.

In diesem Abschnitt soll in Anlehnung an zwei relevante wissenschaftliche Artikel ein Modell erarbeitet werden,
um die \guillemotleft Anfälligkeit\guillemotright des europäischen Flugnetzwerkes auf sich kumulierende Verspätungen zu untersuchen.

\subsubsection{Modell}
\label{subsubsec:model}

Zunächst muss ein Modell erstellt werden, welches sich zur Simulation einer zeitlichen Entwicklung der Verspätungen eignet.
Q. H. Anh Tran und Akira Namatame \cite{anh-tran-worldwide-aviation-network} analysieren in ihrem Artikel das Verhalten
der Effizienz des weltweiten Flugnetzwerkes bei Totalausfall einzelner Flughäfen (Streiks, Kriminalfälle, …) anhand eines vereinfachten und ungerichteten Graphs.
Sie beschreiben zunächst das Netzwerk anhand einer NxN-Matrix, in der jede Verbindung zwischen Flughafen $ i $ und Flughafen $ j $
mit dem Gewicht $ w_{ij} $ eingetragen ist.
Das Gewicht ist die Summe aller Verbindungen zwischen den beiden Flughäfen dar.
Finden beispielsweise vier Flüge von Zürich nach London am Tag statt, so beträgt das Gewicht $ w_{Zuerich London}$ 4.
Im Weiteren wird jedem Flughafen eine Initiallast $L_{i}$ sowie eine Maximalkapazität $ C_{i} = \alpha L_{i}$ zugeordnet.
Die Last errechnet sich aus der Summe aller \guillemotleft effizientesten\guillemotright Pfade, die über den Flughafen $i$ führen.
Dies entspricht ähnlich der \guillemotleft Betweenness Centrality\guillemotright einer Art \guillemotleft Effizienzzentralität\guillemotright.
Im Artikel wird die Nähe zwischen zwei Flughäfen nicht nach dem geografischen Merkmal definiert, sondern nach dem Umkehrwert der Anzahl Direktflügen,
die zwischen den Flughäfen stattfinden ($ p_{ij} = \frac{1}{w_{ij}} $).
Als Distanz zwischen zwei entfernten Flughäfen gilt: $ d_{ij} = \sum_{n=0}^{^k-1}\frac{1}{w_{i_{n}i_{n + 1}}}$.
Die Effizienz ist dann schliesslich: $ e_{ij} = \frac{1}{d_{ij}}$.

Für die vorliegenden Fragestellungen soll das Konzept der Effizienz aus dem Modell von\cite{anh-tran-worldwide-aviation-network} als Grundlage herangezogen werden.
Für die weiteren Details eignet sich aber das Model von Fleurquin und Ramasco besser\cite{fleurquin-ramasco}.
Grundsätzlich funktioniert die Simulation so, dass durch diverse Ereignisse initial Verspätungen entstehen, die durch die
Flugverbindungen an die Zielflughäfen weitergericht werden.
Ein Flughafen wird somit mit Verspätungszeit \guillemotleft aufgeladen\guillemotright und gibt diese Zeit teilweise an die weiteren Abflüge weiter.

Dazu wird das Modell wie folgt aufgebaut und simuliert:

\begin{itemize}
    \item Vorbereitend wird für sämtliche Flughäfen die Maximalkapazität berechnet. Ähnlich \cite{anh-tran-worldwide-aviation-network} wird davon ausgegangen, dass das Netzwerk im Normalzustand nahe an der Maximalkapazität arbeitet. Als Grundlage wird daher die maximale Anzahl Flugbewegungen (Abflüge und Ankünfte), die ein Flughafen pro Stunde abfertigt verwendet. Die Maximalkapazität wird mit dem Faktor $\alpha = 1.2$ berechnet.
    \item Die Simulation beginnt um Mitternacht, 0:00 Uhr. Ein Zeitschritt beträgt eine Minute. Für eine komplette Simulation (ein Tag) werden also 1440 Zustände berechnet.
    \item Für jeden Zeitschritt werden sämtliche zu diesem Zeitpunkt startenden Flüge betrachtet. Starten von einem bestimmten Flughafen mehr Flüge als seine Maximalkapazität erlaubt, wird exponentiell zum Mehraufwand eine kapazitätsbedingte Verspätung generiert.
    \item Wurden dem Abflughafen bereits Verspätungen von ankommenden Flügen deponiert, so werden diejenigen Verspätungen an die Abflüge weitergereicht, deren Entstehungszeitpunkt addiert zum Betrag der Verspätung sowie der Bodenzeit und etwaiger Kapazitätsverspätung den jeweiligen Abflugzeitpunkt übersteigen. Verspätungen, die weit vor einem Abflug entstanden sind, haben somit keinen Einfluss auf den effektiven Abflugzeitpunkt. Unter allen in Frage kommenden Verspätungen wird anschlissend diejenige zu einem gewissen Anteil an die Flugzeit als Verspätung addiert, die zum Abflugzeitpunkt die grösste Überlappung erzeugt.
    \item Für jeden Flug wird eine En-Route-Verspätung von -15 bis +15 Minuten generiert. Damit sollen einerseits Zeitgewinne durch Abkürzungen und höhere Leistung und andererseits Verspätungen durch Wetterbedingungen o.ä. abgebildet werden.
    \item Flughäfen \guillemotleft erholen\guillemotright sich linear von den Verspätungen. Eine Stunde angesammelter Verspätung wird in einem Zeitraum von vier Stunden komplett abgebaut. Dies widerspiegelt die Massnahmen der Flughäfen, um Verspätungen entgegenzuwirken. Etwa durch effizientere Arbeit oder Bevorzugung gewisser verspäteter Flüge auf Kosten anderer Verbindungen.
    \item Hat ein Flug Verspätung, so wird seine Ankunftszeit angepasst. Dies hat wiederum Auswirkungen auf die Last bzw. mögliche Kapazitätsengpässe beim Zielflughafen.
    \item Als durchschnittliche \guillemotleft Turn around\guillemotright Zeit werden 45 Minuten berechnet.
    \item Der Anteil an Passagieren, der auf einen Anschlussflug umsteigt, wird mit 40\% berechnet.
\end{itemize}

Eine Validierung und Präzisierung dieses Modells würde den zeitlichen Umfang der Arbeit übertreffen.
Die Berechnungen des Modells dienen somit lediglich zur qualitativen Analyse des Netzwerks.

Das Resultat eines ersten Durchlaufs der Simulation ist in Abb. \ref{fig:first-simulation-run} ersichtlich.
Darin ist ein starker Anstieg der gesamteuropäischen Verspätung hin zur Mittagszeit zu beobachten, der sich nach der
Mittagszeit genau so schnell wieder abbaut.
Eine Visualisierung der Anzahl Abflüge und Ankünfte (als Summe Anzahl Flugbewegungen) innerhalb Europas macht den
Hintergrund dieser Verspätungsentwicklung deutlich (siehe Abb. \ref{fig:arr-dep-europe}).
Interessanterweise hat eine Erhöhung des Kapazitätsfaktors keinerlei Auswirkung auf die Verspätungsentwicklung.
In zweiter Hinsicht liegt dies jedoch auf der Hand, da die Hauptursache der Verspätungen offensichtlich darin liegt, dass
verspätete Flüge abgewartet werden müssen.
Dies ist auch der Fall, wenn der Flughafen doppelt so viele Landungen zulassen würde.
Deswegen holt ein Flug seine Verspätung allerdings nicht auf (allenfalls minim durch eine frühere Landung ohne Warteschlaufe).

Umgekehrt stellt sich nun die Frage, wie sich die entwicklung der Verspätung in Abhängigkeit der Netzwerkbeschaffenheit ändert.
Im zuvor erwähnten NZZ-Artikel wird von einer Wachstumsrate der Flüge von jährlich 4\% ausgegangen.
In einer weiteren Simulation wird das empirische Netzwerk einem entsprechenden Wachstum für fünf, zehn, 15 beziehungsweise 20 Jahre ausgesetzt.
Das Netzwerk wird dabei jeweils um die Zeit von 5:00 Uhr morgens bis 20:00 Uhr abends mit dem Wachstum entsprechenden zusätzlichen Verbindungen ausgestattet.

Abbildung \ref{fig:sim-after-growth} zeigt die Simulationen nach verschiedenen Wachstumsphasen.
Interessant ist dabei, dass sich die allgemeine Erscheinung der Kurve nicht verändert.
Allenfalls lässt sich eine Verkürzung der Plateaus zwischen 900 und 1200 Minuten feststellen.
Dies könnte darin begründet sein, dass sich Verspätungen aufgrund der höheren Flugbewegungsfrequenz langsamer abbauen
können.

Wie verhält sich die Verspätungsentwicklung in einem zufälligen Netzwerk?
Zu diesem Zweck wird das empirische Flugnetzwerk in ein zufälliges Netzwerk transformiert.
Dabei werden die Flughäfen als Knoten beibehalten, die Verbindungen werden allerdings zufällig zwischen allen
Knoten verteilt.
Abbildung \ref{fig:total-delay-random} zeigt eindrücklich den Unterschied in der Entwicklung der Verspätungen.
Das Zufallsnetzwerk scheint die Weiterverbreitung der Verspätungen besser zu absorbieren als das natürliche Netzwerk.
Dies könnte damit erklärt werden, dass das Potenzial von \guillemotleft Flaschenhälsen\guillemotright in einem
zufälligen Netzwerk weitaus geringer ist als in skalenfreien Netzwerken.
Sind Hubs vorhanden, so ist die Wahrscheinlichkeit gross, dass viele Flüge zur selben Zeit an einem Flughafen ankommen
und möglicherweise einen starken Anstieg der Gesamtverspätung mit sich bringen.
Gleichzeitig wird diese Verspätung mit den Anschlussflügen an sehr viele weitere Flughäfen verteilt.
Anders im Zufallsnetzwerk: Zwar ist die Summe aller Verspätungen der europäischen Flughäfen anfangs grösser, weil
die Verspätung von mehr Flughäfen einzeln gezählt wird als beim natürlichen Netzwerk.
Später wächst die Verspätung allerdings sehr wenig, da die Verspätungen auf mehr Flughäfen verteilt werden können und
weniger Anschlussflüge von einem verspäteten Flug abhängig sind.

\begin{figure}
    \centering
    \includegraphics[width=0.75\linewidth]{images/total-delay-first-simulation-run.pdf}
    \caption{Simulation der gesamteuropäischen Verspätung}
    \label{fig:first-simulation-run}
\end{figure}

\begin{figure}
    \centering
    \includegraphics[width=0.75\linewidth]{images/arr_dep_europe.pdf}
    \caption{Anzahl Flugbewegungen}
    \label{fig:arr-dep-europe}
\end{figure}

\begin{figure}
    \centering
    \subfigure[Simulation nach Wachstum um 5 Jahre]{\includegraphics[width=0.45\linewidth]{images/total_delay_5yrs.pdf}}
    \subfigure[Simulation nach Wachstum um 10 Jahre]{\includegraphics[width=0.45\linewidth]{images/total_delay_10yrs.pdf}}
    \subfigure[Simulation nach Wachstum um 15 Jahre]{\includegraphics[width=0.45\linewidth]{images/total_delay_15yrs.pdf}}
    \subfigure[Simulation nach Wachstum um 20 Jahre]{\includegraphics[width=0.45\linewidth]{images/total_delay_20yrs.pdf}}
    \caption{Simulationen nach Wachstum des Netzwerks}
    \label{fig:sim-after-growth}
\end{figure}

\begin{figure}
    \centering
    \includegraphics[width=0.75\linewidth]{images/total_delay_random.pdf}
    \caption{Simulation der gesamteuropäischen Verspätung mit Zufallsnetzwerk}
    \label{fig:total-delay-random}
\end{figure}

Natürlich widerspiegelt ein zufälliges Netzwerk die Realität in keinster Weise, aber es liefert zumindest Anhaltspunkte,
wie die zukünftige Entwicklung des natürlichen Netzwerkes optimal gestaltet werden könnte.

In diesem Zusammenhang kann folgende weitere Hypothese aufgestellt werden: \guillemotleft Die Gesamtverspätung des
europäischen Flugnetzwerkes nimmt weniger zu, wenn beim Ausbau des Netzwerks vor allem Direktverbindungen
bevorzugt werden, als wenn Verbindungen gemäss der bereits vorhandenen Hub-Struktur angehäuft werden.\guillemotright.

Um dies zu untersuchen, soll die Simulation auf zwei Arten angepasst werden:
\begin{enumerate}
    \item Das Wachstum des Netzwerks über 10 Jahre bevorzugt Flughafenpaare, zwischen denen bereits eine starke Verbindung (sprich viele tägliche Verbindungen) besteht.
    \item Das Wachstum des Netzwerks über 10 Jahre bevorzugt Flughafenpaare, zwischen denen bereits eine starke Verbindung besteht, stellt stattdessen jedoch Direktverbindungen zwischen einem der beiden Flughäfen und einem direkten Nachbarn des anderen Flughafen her.
\end{enumerate}

Analog zum \guillemotleft Copying Model\guillemotright wird die Verbindungswahrscheinlichkeit für Fall 1 wie folgt beschrieben:

\begin{equation}
    \label{eq:growth-model-1}
    \prod (k_{ij}) = p\frac{1}{N} + (1-p)\frac{w_{ij}}{\sum_{z}w_{iz}}
\end{equation}

Ein bestehender Flughafen $i$ stellt also während des Wachstums zunächst zu jedem anderen Flughafen $j$ mit der Wahrscheinlichkeit
$p \frac{1}{N}$ eine Verbindung her.
Existieren zwischen den Flughäfen $i$ und $j$ hingegen bereits Verbindungen, so steigt die Verbindungswahrscheinlichkeit in Abhängigkeit
der Anzahl $w_{ij}$ dieser bereits bestehenden Verbindungen im Verhältnis zu allen bereits bestehenden Verbindungen ausgehend von Flughafen $i$ (siehe Abb. \ref{fig:pref-attach-hubs}).
Dies soll die Realität beschreiben, wo bereits beliebte Flugrouten durch die steigende Nachfrage in der Kapazität (sprich Anzahl Flüge pro Tag)
ausgebaut werden.

Für den zweiten Fall dient Gleichung \ref{eq:growth-model-1} als Basis, allerdings mit der Änderung, dass bestehende Direktverbindungen \textit{nicht} weiter
ausgebaut werden sollen.
Stattdessen sollen neue Verbindungen zu den Nachbarn dieser bestehenden Direktverbindungen hergestellt werden, wenn zu diesen Nachbarn noch keine
oder nur wenige Direktverbindungen bestehen (siehe Abb. \ref{fig:pref-attach-neighbors}).

\begin{equation}
    \label{eq:growth-model-2}
    \prod (k_{in}) = p\frac{1}{N} + w_{n}\Bigg[ (1-p)\frac{1}{1 + w_{in}^{2}}\frac{w_{ij}}{\sum_{z}w_{iz}} \Bigg], w_{n} = \begin{cases}1,& \text{falls } w_{jn}\geq 1\\ 0, & \text{andernfalls}
    \end{cases}
\end{equation}

\begin{figure}
    \centering
    \subfigure[\label{fig:pref-attach-hubs}]{\includegraphics[width=0.30\linewidth]{images/growth-model-1.pdf}} \hspace{4em}
    \subfigure[\label{fig:pref-attach-neighbors}]{\includegraphics[width=0.30\linewidth]{images/growth-model-2.pdf}}
    \label{fig:pref-attach}
    \caption{Bevorzugung von Hubs (a) sowie von Nachbarn von Hubs (b)}
\end{figure}


Gleichung \ref{eq:growth-model-2} erhöht die Verbindungswahrscheinlichkeit nur dann, wenn zwischen den Hub-Flughafen $j$ und dessen Nachbar $n$ bereits mindestens eine Verbindung besteht.
Ist dies nicht der Fall, ist eine Direktverbindung von $i$ nach $n$ nicht sinnvoll, weil Passagiere bereits heute nicht von $i$ nach $n$ reisen.
Die Wahrscheinlichkeit, dass $i$ eine Verbindung nach $n$ herstellt unterliegt dann also einem gleichmässigen Zufall.

Besteht allerdings bereits mindestens eine Verbindung zwischen Hub und Nachbar, so ist es umso wahrscheinlicher, dass eine Direktverbindung von $i$ nach $n$ aufgebaut wird,
je weniger Direktverbindungen zwischen diesen Flughäfen bereits bestehen ($ \frac{1}{1 + w_{in}^{2}}$).

Dies soll das Bestreben des Wachstums widerspiegeln, Lasten von Hubs abzuwenden und stattdessen auf noch wenig entwickelte Direktrouten zu verteilen.

Abbildung \ref{fig:delay} zeigt die Simulationen nun im Vergleich.
Auffällig ist wiederum, dass sich die allgemeine Charakteristik der täglichen Entwicklung nicht merklich verändert.
Allerdings reduziert sich die maximale Verspätung auf rund 30000 Minuten gegenüber einer zufälligen Routenentwicklung über 10 Jahre.
Nach dem Wachstumsmodell 1, bei dem neue Verbindungen fokussiert auf bereits starke Verbindungen entstehen, erhöht sich die maximale Verspätung hingegen auf rund 40000 Minuten.
Um die Simulation in nützlicher Frist mehrfach zu wiederholen, wurden das Netzwerk zufällig in der Kantenzahl reduziert.
In der zusammenfassenden Abbildung \ref{fig:delay-all} ist nach mehreren Simulationsgängen stets zu erkennen, dass das Modell für die Direktverbindungen tendentiell zu geringeren Gesamtverspätungen führt.

\begin{figure}
    \centering
    \subfigure[Simulation nach Wachstum um 10 Jahre mit Wachstumsmodell 2]{\includegraphics[width=0.45\linewidth]{images/total_delay_neighbor10yrs.pdf}}
    \subfigure[Simulation nach Wachstum um 10 Jahre mit zufälligem Wachstum]{\includegraphics[width=0.45\linewidth]{images/total_delay_10yrs.pdf}}
    \subfigure[Simulation nach Wachstum um 10 Jahre mit Wachstumsmodell 1]{\includegraphics[width=0.45\linewidth]{images/total_delay_hub_10yrs.pdf}}
    \caption{Simulationen nach Wachstum des Netzwerks}
    \label{fig:delay}
\end{figure}

\begin{figure}
    \centering
    \includegraphics[width=0.75\linewidth]{images/total_delay_all.pdf}
    \caption{Verspätung nach Wachstum über 10 Jahre}
    \label{fig:delay-all}
\end{figure}


\subsection{Ausblick}
Diese Arbeit berührt einige Überlegungen zum weltweiten Flugnetzwerk und insbesondere zum europäischen Flugnetzwerk nur ansatzweise.
Als nächsten Schritt müsste das Simulationsmodell validiert werden, um die Richtigkeit der Prognosen zu überprüfen und verbessern zu können.
Anschliessend wäre die Weiterentwicklung der Wachstumsmodelle interessant.
In dieser Arbeit werden diese lediglich unter dem Gesichtspunkt der Verspätungsentwicklung betrachtet.
In diesem Zusammenhang wäre es wichtig, diese Resultate anderen Charakteristiken des Netzwerks gegenüberzustellen.
Beispielsweise wurde die Robustheit nur kurz angesprochen.
Daher wäre es aufschlussreich, die Entwicklung der Verspätungen bei verschiedenen Netzwerkbeschaffenheiten beispielsweise der Robustheit bei
Totalausfällen oder der Kostenentwicklung für den Ausbau eines bestimmten Netzwerks gegenüberzustellen.
