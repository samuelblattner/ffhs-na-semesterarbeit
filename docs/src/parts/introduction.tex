\section{Übersicht und Inhalt}
\label{sec:contents}

Die Vorliegende Arbeit beschäftigt sich mit der Analyse des weltweiten Flugnetzwerkes im Rahmen der Semesterarbeit für
das Modul Network-Analysis an der Fernfachhochschule Schweiz.

Im ersten Teil wird zunächst beschrieben, wie das Netzwerk aus verschiedenen öffentlich verfügbaren Datensätzen
aufgebaut wird.
Im zweiten Teil werden einige Grundcharakteristiken des Netzwerks beschrieben und analysiert.
Der letzte Teil legt den Fokus auf das europäische Netzwerk und geht der Frage nach, wie sich Verspätungen im Netzwerk
ausbreiten.
Dazu wird ein einfaches Modell erstellt, mit dem die Verspätungsentwicklung im Netzwerk simuliert werden kann.
Die Erwartung an diese Simulationen ist es, qualitative Prognosen über die Entwicklung der Verspätungen in Abhängigkeit
unterschiedlicher Präferenzen bem zukünftigen Ausbau des Streckennetzes zu machen.

Zur besseren Übersicht werden die kompletten Listings der Scripts, die für diese Arbeit erstellt wurden, nicht im
vorliegenden Dokument aufgeführt.
Stattdessen werden vereinzelt relevante Auszüge der Scripts abgebildet.
Die vollständigen Scripts sind öffentlich im Github-Repository verfügbar unter https://github.com/samuelblattner/ffhs-na-semesterarbeit.